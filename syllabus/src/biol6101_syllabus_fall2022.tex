\documentclass[12pt, notitlepage]{article}   	% use "amsart" instead of "article" for AMSLaTeX format
\usepackage{geometry}                		% See geometry.pdf to learn the layout options. There are lots.
\geometry{a4paper}                   		% ... or a4paper or a5paper or ... 
%\geometry{landscape}                		% Activate for rotated page geometry
\usepackage[parfill]{parskip}    		% Activate to begin paragraphs with an empty line rather than an indent
\usepackage{graphicx}				% Use pdf, png, jpg, or eps§ with pdflatex; use eps in DVI mode
								% TeX will automatically convert eps --> pdf in pdflatex

\usepackage{hyperref}
		
%SetFonts

\usepackage[T1]{fontenc}
\usepackage[utf8]{inputenc}

\usepackage{tgbonum}

%SetFonts

\title{
	\textbf{
		BIOL 6101-029
	} \\
	\large Seminar: Plant physiological acclimation to global change \\
	\large Fall 2022
}

\date{\vspace{-5ex}}

\begin{document}

{\fontfamily{phv}\selectfont %select helvetica (code = phv)

\maketitle

\section{Course Description}
Students in this course will read and discuss classical and recent literature on
plant physiological acclimation to global change. Specific articles and topics
will be chosen based on student interests.

\subsection{Class Time and Location}
Day and time: TBD

Experimental Sciences Building II (ESB II) Room 409 or 
or otherwise agreed upon.

\subsection{Instructor}
Dr. Nick Smith \par
ESB II Room 402D \par
806-834-7363 \par
nick.smith@ttu.edu \par
\textit{Meetings by appointment}

\subsection{Recommended Texts}
Plant Physiological Ecology (2nd Edition; 2008) by Lambers, Chapin, and Pons \par
The book can be accessed from Springer here: 
\url{https://www.springer.com/us/book/9780387783406}. Click on "Access this title on 
SpringerLink." It can also be accessed through the TTU library. \par
Plant Physiology and Development (6th Edition) by Taiz, Ziegler, Moller, and Murphy

\section{Mode of Instruction}
All instruction will be done face-to-face unless the university directs classes be taught online (see next section).

\section{Contingency Statement}
This course is being taught primarily in the face-to-face learning mode. 
The University will continue to monitor CDC, State, and TTU System guidelines in 
continuing to manage the campus implications of COVID-19. 
Any changes affecting class policies or delivery modality will be in accordance with 
those guidelines and announced as soon as possible. 
If Texas Tech University campus operations are required to change because of health 
concerns related to the COVID-19 pandemic, it is possible that this course will move 
to a fully online delivery format. 
Should that be necessary, students will need to have access to the Internet, a webcam, 
and microphone for remote delivery of the class. 

\section{Course Materials}
All course materials, including lecture slides, readings, activities, and code 
will be posted to a GitHub repository for the course.
The primary repository address is
\url{https://github.com/SmithEcophysLab/biol6101_fall2022}.

\section{Learning Objective}
This course will broadly focus on understanding plant physiological acclimation
responses to changes in environmental conditions, with a focus on global changes
that are expected due to ongoing anthropogenic activities.
Class activities will be based on discussion and dissemination of ideas, 
including classic and recent scientific literature. 
Topics will be flexible and modified to match student interests where possible.

\section{Attendance Policy}
Attendance is strongly recommended. 
The course assessments will be done during class (see below).

\section{Course Assessment}
\subsection{\textit{Participation and Engagement}}
Being an active and engaged participant in the class will benefit your understanding
of material as well as your peers'. Examples include asking questions, providing feedback,
and facilitating discussion. Participation and engagement of each student will be monitored
during each class period and will constitute the only assessment.

\section{Grading}
Participation and Engagement: 100\% \par

\section{Grading Scale}
A: $\geq$ 90\% \par
B: 80 – 90\% \par
C: 70 – 80\% \par
D: 60 – 70\% \par
F: $\leq$ 59.9\% \par

\section{Missing In-class Activities}
Students will be required to be in class to receive participation and engagement points. 
Please read below if class is to be missed due to an officially approved trip, illness, or special circumstance:

\subsection{Officially Approved Trips}
The person responsible for a student representing the University on officially 
approved trips should notify the instructor of the departure and return schedules in advance. 
For other University-approved curricular and extracurricular activities, 
the instructor must be presented with verifiable documentation prior to the first absence. 
The student will not be penalized for the absence but is responsible for the material missed.

\subsection{Illness Based Absence Policy}
If at any time during this semester you feel ill, in the interest of your own 
health and safety as well as the health and safety of your instructors and classmates, 
you are encouraged not to attend face-to-face class meetings or events. 
Please notify your instructors as soon as possible to ensure your absence for 
illness will be excused. 
You are strongly encouraged to visit with either Student Health Services at (806) 743-2848 
or your health care provider. 
A “return to school” note from your provider will be required to return to class. 
You will still be responsible to complete within a week of returning to class any 
assignments, quizzes, or exams you miss because of illness.

\subsection{Special Circumstance Absence}
There may be special circumstances that render missing class unavoidable.
If this arises, please let Dr. Smith know of the situation as soon as possible,
so that the loss of point due to the absence can be discussed.

\section{TTU COVID-19 Policy Reminders}
\begin{itemize}
	\item{Although COVID-19 vaccinations are not mandated, Texas Tech strongly 
	recommends that all students be vaccinated and receive a booster when eligible.
	The vaccines are safe and effective.}
	\item{Please visit the university’s coronavirus (COVID-19) page for additional 
	information about on-campus vaccination and testing schedules, reporting a 
	positive test result, and submitting vaccination records: 
	\url{https://www.depts.ttu.edu/communications/emergency/coronavirus/}.}
	\item{Face masks are strongly encouraged in classrooms and other public indoor 
	settings on campus, including the Student Wellness Center.}
	\item{If you are sick or not feeling well, you should stay at your place of 
	residence and wear a mask when around others.  Do not attend class, work, or social functions. 
	If you brought a COVID-19 home test kit, please use it to determine whether you are 
	positive for the virus.}
	\item{For students living in an on-campus residence, a limited number of tests 
	are available from Community Advisors in residence halls. Please reach out to 
	yours virtually to request one. }
	\item{Students can also be tested at an on-campus site.}
	\item{Students who meet the qualifications can contact Student Health Services to 
	schedule an appointment to be tested. 
	All students in university housing should develop an action plan in the event 
	they are required to self-isolate due to a positive COVID-19 diagnosis. 
	This plan should include a location to complete the self-isolation, access to 
	groceries/meal delivery, access to necessary medications, numbers of emergency 
	contacts, and contact information for their preferred healthcare provider.}
	\item{All students (both vaccinated and unvaccinated) who have been identified as 
	having a known exposure to a COVID-19 positive person should follow CDC guidance 
	(\url{https://www.cdc.gov/coronavirus/2019-ncov/your-health/quarantine-isolation.html}), 
	which says:
		\begin{itemize}
			\item{If you are exposed but not vaccinated or up to date on vaccinations and boosters:
				\begin{itemize}
					\item{Quarantine for at least five days.}
					\item{Wear a well-fitting (preferably N95 or KN95) mask if you must be around others.}
					\item{Do not travel.}
					\item{Get tested at least five days after exposure.}
				\end{itemize}
			}
			\item{If you are exposed and are up to date on vaccinations and boosters:
				\begin{itemize}
					\item{No quarantine is necessary unless you develop symptoms.}
					\item{Get tested at least five days after exposure.}
				\end{itemize}
			}
			\item{If you exposed and have had confirmed COVID-19 within the past 90 days:
				\begin{itemize}
					\item{No quarantine is necessary unless you develop symptoms.}
				\end{itemize}
			}
		\end{itemize}
	}
	\item{Self-isolation for five days is required for all students 
			(vaccinated or unvaccinated) who test positive for COVID-19. 
			After the five-day isolation period, if the student is asymptomatic or 
			their symptoms are resolving (fever free without the use of fever reducing 
			medication for 24 hours), they may return to class/activities but should 
			wear a face mask for an additional five days.}
	\item{Students who are positive should report the result. 
			This generates a letter that you can provide to your professors and 
			instructors, notifying them of your positive diagnosis.}
\end{itemize}

\section{Special Considerations}
\subsection{Accommodations for Disabilities}
Any student who, because of a disability, may require special arrangements to meet the 
course requirements should contact the instructor as soon as possible to make any 
necessary arrangements. 
Students should present appropriate verification from Student Disability Services 
during the instructor’s office hours. 
Please note that instructors are not allowed to provide classroom accommodations 
to a student until appropriate verification from Student Disability Services has been provided. 
For additional information, please contact the Student Disability Services office in 130 
Weeks Hall or call 806-742-2405.

\subsection{Religious Holy Days}
“Religious holy day” means a holy day observed by a religion whose places of worship 
are exempt from property taxation under Texas Tax Code §11.20. 
A student who intends to observe a religious holy day should make that intention known 
in writing to the instructor prior to the absence. 
A student who is absent from classes for the observance of a religious holy day shall be 
allowed to take an examination or complete an assignment scheduled for that day within a 
reasonable time after the absence. 
A student who is excused may not be penalized for the absence; however, the instructor 
may respond appropriately if the student fails to complete the assignment satisfactorily.

\section{TTU Resources for Discrimination, Harassment, and Sexual Violence}
Texas Tech University is committed to providing and strengthening an educational, 
working, and living environment where students, faculty, staff, and visitors are 
free from gender and/or sex discrimination of any kind. 
Sexual assault, discrimination, harassment, and other Title IX violations are 
not tolerated by the University. 
Report any incidents to the Office for Student Rights & Resolution, 
(806)-742-SAFE (7233), or file a report online at \url{https://www.depts.ttu.edu/titleix/}. 
Faculty and staff members at TTU are committed to connecting you to resources on campus. 
Some of these available resources are: TTU Student Counseling Center, 806-742-3674, 
\url{https://www.depts.ttu.edu/scc/} (provides confidential support on campus). 
TTU Student Counseling Center 24-hour Helpline, 806-742-5555, 
(assists students who are experiencing a mental health or interpersonal violence crisis; 
if you call the helpline, you will speak with a mental health counselor). 
Voice of Hope Lubbock Rape Crisis Center, 806-763-7273, \url{voiceofhopelubbock.org} 
(24-hour hotline that provides support for survivors of sexual violence). 
The Risk, Intervention, Safety and Education (RISE) Office, 806-742-2110, 
\url{https://www.depts.ttu.edu/rise/} (provides a range of resources and 
support options focused on prevention education and student wellness). 
Texas Tech Police Department, 806-742-3931, \url{http://www.depts.ttu.edu/ttpd/} 
(to report criminal activity that occurs on or near Texas Tech campus). 

\section{LGBTQIA}
Please contact the Office of LGBTQIA, Student Union Building Room 201, 806-742-5433, 
\url{www.lgbtqia.ttu.edu}. 
Within the Center for Campus Life, the Office serves the Texas Tech community 
through facilitation and leadership of programming and advocacy efforts. 
This work is aimed at strengthening the lesbian, gay, bisexual, transgender, queer, 
intersex, and asexual (LGBTQIA) community and sustaining an inclusive campus that 
welcomes people of all sexual orientations, gender identities, and gender expressions. 

\section{Classroom Civility}
Texas Tech University is a community of faculty, students, and staff that enjoys 
an expectation of cooperation, professionalism, and civility during the conduct of all 
forms of university business, including the conduct of student–student and student–faculty 
interactions in and out of the classroom. 
Further, the classroom is a setting in which an exchange of ideas and creative thinking 
should be encouraged and where intellectual growth and development are fostered. 
Students who disrupt this classroom mission by rude, sarcastic, threatening, abusive or 
obscene language and/or behavior will be subject to appropriate sanctions according to 
university policy.  Likewise, faculty members are expected to maintain the highest standards of professionalism in all interactions with all constituents of the university (www.depts.ttu.edu/ethics/matadorchallenge/ethicalprinciples.php).

\section{Academic Integrity}
Academic integrity is taking responsibility for one’s own class and/or course work, 
being individually accountable, and demonstrating intellectual honesty and ethical behavior. 
Academic integrity is a personal choice to abide by the standards of intellectual honesty 
and responsibility. 
Because education is a shared effort to achieve learning through the exchange of ideas, 
students, faculty, and staff have the collective responsibility to build mutual trust and respect. 
Ethical behavior and independent thought are essential for the highest level of academic 
achievement, which then must be measured. 
Academic achievement includes scholarship, teaching, and learning, all of which are shared endeavors. 
Grades are a device used to quantify the successful accumulation of knowledge through learning. 
Adhering to the standards of academic integrity ensures grades are earned honestly. 
Academic integrity is the foundation upon which students, faculty, and staff build their 
educational and professional careers. [Reference: Texas Tech University Quality 
Enhancement Plan, Academic Integrity Task Force, 2010].

\section{Schedule of Topics by Week}
29/08/22 – Introductions, semester planning, and goals \par
05/09/22 – Physiology and ecology basics (as needed) \par
12/09/22 – Acclimation TBD \par
19/09/22 – Acclimation TBD \par
26/09/22 – Acclimation TBD \par
03/10/22 – Acclimation TBD \par
10/10/22 – Acclimation TBD \par
17/10/22 – Acclimation TBD \par
24/10/22 – Acclimation TBD \par
31/10/22 – Acclimation TBD \par
07/11/22 – Acclimation TBD \par
14/11/22 – Acclimation TBD \par
21/11/22 – Acclimation TBD \par
28/11/22 – Acclimation TBD \par

} %end font selection

\end{document} 
